\begin{enumerate}
  \item [1.] Написать на любом ЯВУ программу моделирования прогнозирующей линейной ИНС.
  Для тестирования использовать функцию:

$$
y = a * sin(b * x) + d
$$

Варианты заданий приведены в следующей таблице:

\begin{tabular}{ | c | c | c | c | c | }
  \hline
  № варианта & a	& b	& d	& Кол-во входов ИНС\\
  \hline
  1	  & 1 & 5 & 0.1 & 3 \\
  2	  & 2 & 6 & 0.2 & 4 \\
  3	  & 3 & 7 & 0.3 & 5 \\
  4	  & 4 & 8 & 0.4 & 3 \\
  5	  & 1 & 9 & 0.5 & 4 \\
  6	  & 2 & 5 & 0.6 & 5 \\
  7	  & 3 & 6 & 0.1 & 3 \\
  8	  & 4 & 7 & 0.2 & 4 \\
  9	  & 1 & 8 & 0.3 & 5 \\
  10	& 2 & 9 & 0.4 & 3 \\
  11	& 3 & 5 & 0.5 & 4 \\
  \hline
\end{tabular}

Обучение и прогнозирование производить на 30 и 15 значениях соответственно табулируя функцию с шагом 0.1. Скорость обучения выбирается студентом самостоятельно, для чего моделирование проводится несколько раз для разных $\alpha$. Результаты оцениваются по двум критериям - скорости обучения и минимальной достигнутой ошибке. Необходимо заметить, что эти критерии в общем случае являются взаимоисключающими, и оптимальные значения для каждого критерия достигаются при разных $\alpha$. 

  \item [2.] Результаты представить в виде отчета содержащего:
  
  \begin{enumerate}
    \item Титульный лист
    \item Цель работы
    \item Задание
    \item Результаты обучения: таблицу  со столбцами: эталонное значение, полученное значение, отклонение; график изменения ошибки в зависимости от итерации.
    \item Результаты прогнозирования: таблицу  со столбцами: эталонное значение, полученное значение, отклонение.
    \item Выводы по лабораторной работе.
  \end{enumerate}
  
  Результаты для пунктов 3 и 4 приводятся для значения $\alpha$, при котором достигается минимальная ошибка. В выводах анализируются все полученные результаты.
  
\end{enumerate}

\section*{Контрольные вопросы}

\begin{enumerate}
  \item ИНС какой архитектуры Вы использовали в данной работе? Опишите принцип построения этой ИНС.
  \item Как функционирует используема Вами ИНС?
  \item Опишите (в общих чертах) алгоритм обучения Вашей ИНС.
  \item Как формируется обучающая выборка для решения задачи прогнозирования?
  \item Как выполняется многошаговое прогнозирование временного ряда?
  \item Предложите критерий оценки качества результатов прогноза.
\end{enumerate}
